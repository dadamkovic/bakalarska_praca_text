%% !TeX spellcheck = sk_SK
%\chapter{Úvod}
%
%Dvojkolesové balansujúce roboty predstavujú v rámci robotických systémov zaujímavú skupinu robotov, ktorá účelovo predstavuje určitý medzikrok medzi klasickými, inherentne stabilnými systémami na kolesách a bipedálnymi robotmi napodobňujúcimi spôsob chôdze ľudí.
% 
%Avšak, kým klasické viac-kolesové roboty sa vyznačujú výbornými vlastnosťami ako v oblasti stability tak aj rýchlosti pohybu, vo všeobecnosti sú väčšie a zložitejšie (a teda aj drahšie) ako dvojkolesové roboty navrhnuté pre rovnaký účel. Výbornou ukážkou je napríklad populárny Segway, pri ktorom výrobca efektívne využil balansovanie na dvoch kolesách, bez obmedzenia užitočnosti produktu.   
%
%Naproti tomu návrh, naprogramovanie a skonštruovanie robotických systémov s umelými nohami je v súčastnosti stále problém vyžadujúci nasdanie komplexných, ťažko naladiteľných regulátorov. Tieto systémy sa tak väčšinou javia ako príliš drahé, nespoľahlivé a pomalé na nasadenie v praxi. 
%
%My sa v tejto práci budeme zaoberať návrhom, analýzou a konštrukciou modulárneho, balansujúceho robota, na ktorom demonštrujeme vyššie popísané vlastnosti.  

% !TeX spellcheck = sk_SK
\chapter{Úvod}

Dvojkolesové balansujúce roboty predstavujú v rámci robotických systémov zaujímavú skupinu robotov, ktorá účelovo predstavuje určitý medzikrok medzi klasickými, inherentne stabilnými systémami na kolesách a bipedálnymi robotmi napodobňujúcimi spôsob chôdze ľudí.
 
Klasické viackolesové roboty sa vyznačujú výbornými vlastnosťami ako v oblasti stability tak aj rýchlosti pohybu, ale vo všeobecnosti sú väčšie a zložitejšie (a teda aj drahšie) ako dvojkolesové roboty navrhnuté pre rovnaký účel. Výbornou ukážkou je napríklad populárny Segway, pri ktorom výrobca efektívne využil balansovanie na dvoch kolesách bez obmedzenia užitočnosti produktu.   

Na druhej strane návrh, naprogramovanie a skonštruovanie robotických systémov s umelými nohami je v súčasnosti stále problém vyžadujúci nasadenie komplexných, ťažko naladiteľných regulátorov. Tieto systémy sa tak väčšinou javia ako príliš drahé, nespoľahlivé a pomalé na praktické použitie. 

My sa v tejto práci budeme zaoberať návrhom, analýzou a konštrukciou modulárneho, balansujúceho robota.  
