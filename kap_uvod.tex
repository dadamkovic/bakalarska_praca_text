% !TeX spellcheck = sk_SK
\chapter{Úvod}


Dvojkolesové balansujúce roboty predstavujú vrámci robotických systémov zaujímavú skupinu robotov, ktorá účelovo predstavuje kompromis medzi klasickými, inherentne stabilnými systémami na kolesách a bipedálnymi robotmi napodobňujúcimi spôsob chôdze ľudí.
 
Avšak kým klasické viac-kolesové roboty sa vyznačujú výbornými vlastnosťami ako v oblasti stability tak aj rýchlosti pohybu, takéto riešenia je náročné implementovať ak existuje požiadavka aby sa robotický systém veľkostne a tvarovo podobal človeku. S rastúcou výškou systému je následne potrebné zväčšovať aj veľkosť podvozku a udržiavať ťažisko čo najbližšie zemi aby systém zostal pri pohybe stabilný. Takýto robot sa ale stáva masívnejším, čo môže predstavovať problém hlavne pri pohybe v interiéroch.

Naproti tomu návrh, naprogramovanie a implementovanie robotických systémov s umelými nohami je v súčastnosti stále problém. Tieto systémy sa tak často javia ako príliš pomalé, ťažkopádne a nespoľahlivé na nasadenie v blízkosti ľudí. 

Koncept balansujúceho robota v sebe spája ako výhody vyplývajúce z pohybu na kolesách, tak aj možnosť navrhnúť konštrukčne jednoduchý systém, ktorý do značnej miery kopíruje anatómiu človeka. Toto riešenie sa javí ako značne výhodné hlavne v rýchlo sa rozvíjajúcej oblasti spoločenských robotov, ktorí sú takto schopní vykonávať mnoho činností v prostrediach navrhnutých pre pohyb ľudí. 

V tejto práci sa budeme zaoberať návrhom, analýzou a konštrukciou práve takéhoto modulárneho, balansujúceho robota, na ktorom demonštrujeme vyššie popísané vlastnosti. 